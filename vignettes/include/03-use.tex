\section{Using the Package}


\subsection{Background}

The input to the n-gram processor must be a single string (character vector of length 
1).  To aid in what could be a repetitive task, the package offers 
the \code{concatenate()} function.  For example:

\begin{lstlisting}[language=inteRactive]
> letters
 [1] "a" "b" "c" "d" "e" "f" "g" "h" "i" "j" "k" "l" "m" "n" "o" "p" "q" "r" "s"
[20] "t" "u" "v" "w" "x" "y" "z"
> library(ngram)
> concatenate(letters)
[1] "abcdefghijklmnopqrstuvwxyz"
> concatenate(concatenate(letters), letters, collapse=" ")
[1] "abcdefghijklmnopqrstuvwxyz a b c d e f g h i j k l m n o p q r s t u v w x y z"
\end{lstlisting}

So if data is coming from multiple files, the simplest way to merge them 
together would be to call

\begin{lstlisting}[language=rr]
x <- readLines("file1")
y <- readLines("file2")

str <- concatenate(x, y)
\end{lstlisting}

The \code{ngram()} function (which does the processing and forms the n-grams) always splits words 
at a 
space. You can preprocess the string with \R's regular expression 
utilities, such as \code{gsub()}, or use the \code{preprocess()} utility in 
the \thispackage package to be able to split at non-spaces for the purpose of n-gram generation (by 
inserting your own beforehand).



\subsection{Package Use and Example}

The general process goes
\begin{enumerate}
  \item Prepare the input string; you may find \code{concatenate()} and 
\code{preprocess()} useful (see the previous subsection).
  \item Process with \code{ngram()}.
  \item Generate new text with \code{babble()}, and/or
  \item[3.5] Extract pieces of the processed ngram data with the \code{get.*()} 
functions.
\end{enumerate}



\subsubsection{Creating}

Let us return to the example sequence of letters from \secref{sec:intro}.  If 
we store this string in \code{x}:

\begin{lstlisting}[language=rr]
x <- "A B A C A B B"
\end{lstlisting}

then the next step is to process with \code{ngram()}:

\begin{lstlisting}[language=rr]
library(ngram)
ng <- ngram(x, n=2)
\end{lstlisting}

Simple as that!  And the tokenization was designed to be extremely fast; see 
\secref{sec:benchmarks} for benchmarks.



\subsubsection{Inspecting}

We can then inspect the sequence:

\begin{lstlisting}[language=inteRactive]
> ng
[1] "An ngram object with 5 2-grams"
\end{lstlisting}

If you don't have too many n-grams, you may want to print all of them by 
calling 
\code{print()} directly, with option \code{full=TRUE}:
\begin{lstlisting}[language=inteRactive]
> print(ng, full=TRUE)
C A 
B {1} | 

B A 
C {1} | 

B B 
NULL {1} | 

A C 
A {1} | 

A B 
A {1} | B {1} | 
\end{lstlisting}

Here we see each 3-gram, followed by its next possible ``words'' and each 
word's frequency of occurrence (occurrence following the given n-gram).  So in 
the above, the first n-gram printed \code{C A} has \code{B} as a next 
possible word, because the sequence \code{C A} is only ever followed by the 
``word'' \code{B} in the input string.  On the other hand, \code{A B} is 
followed by \code{A} once and \code{B} once.  The sequence \code{B B} is 
terminal, i.e. followed by nothing; we treat this case specially.



\subsubsection{Summarizing}

TODO word counter

Once the \code{ngram} representation of the text has been generated, it is very 
simple to get some interesting summary information.  The function 
\code{get.phrasetable()} generates a ``phrasetable'', or more explicitly, a 
table of n-grams, and their frequency and proportion in the text:

\begin{lstlisting}[language=inteRactive]
> get.phrasetable(ng)
   ngrams freq      prop
 1    A B    2 0.3333333
 2    B A    1 0.1666667
 3    C A    1 0.1666667
 4    A C    1 0.1666667
 5    B B    1 0.1666667
\end{lstlisting}


We can perhaps better see the value of this in a more interesting string:

\begin{lstlisting}[language=inteRactive]
text <- concatenate(sample(LETTERS[1:3], size=100, replace=TRUE), collapse=" ")
> text
[1] "A A B B B A C B A B B B B C B C B A C C C B A B C A B B B A C A C B C C A B 
B A A A C A C A C C A B C A A B B B B C A C A C B B B B C B A A C B A B A B B B 
A B C C A A C C A A C B B A A C B B A A B B"
> head(get.phrasetable(ngram(text, n=3)))
  ngrams freq       prop
1  B B B    9 0.09183673
2  A B B    7 0.07142857
3  A C B    6 0.06122449
4  B B A    6 0.06122449
5  B A B    5 0.05102041
6  A A C    5 0.05102041
\end{lstlisting}



\subsubsection{Babbling}

Next, we might want to generate some new strings.  We for this, we use 
\code{babble()}:

\begin{lstlisting}[language=inteRactive]
> babble(ng, 10)
[1] "A C A B B B A C A B "
> babble(ng, 10)
[1] "B B C A B A C A B A "
> babble(ng, 10)
[1] "A C A B A C A B A C "
\end{lstlisting}

This generation includes a random process.  For this, we developed our own 
implementation of MT19937, and so R's seed management does not apply.  To 
specify your own seed, use the \code{seed=} argument:
\begin{lstlisting}[language=inteRactive]
> babble(ng, 10, seed=10)
[1] "A C A B A C A B B B "
> babble(ng, 10, seed=10)
[1] "A C A B A C A B B B "
> babble(ng, 10, seed=10)
[1] "A C A B A C A B B B "
\end{lstlisting}




\subsection{Important Notes About the Internal Representation}

The entirety of the interesting bits of the \thispackage package take place 
outside of \R (completely in \C).  Observe:
\begin{lstlisting}[language=inteRactive]
> str(ng)
Formal class 'ngram' [package "ngram"] with 6 slots
  ..@ str_ptr:<externalptr> 
  ..@ strlen : int 13
  ..@ n      : int 2
  ..@ ng_ptr :<externalptr> 
  ..@ ngsize : int 5
  ..@ wl_ptr :<externalptr> 
\end{lstlisting}

So everything is wrangled up top as an S4 class, and underneath the data is 
stored as 2 linked lists, outside the purview of \R.  This means that, for 
example, that you cannot save the n-gram object with a call to \code{save()}.  
If you do and you shut down and restart \R, the pointers will no longer be 
valid.

Extracting a the data into a native \R data structure is not currently 
possible.  Full support is planned for a later release.  Some pieces can be 
extracted.  At this time, \code{get.ngrams()} and \code{get.string()} are 
implemented, but \code{get.nextwords()} is not.

\begin{lstlisting}[language=inteRactive]
> get.ngrams(ng)
[1] "C A" "B A" "B B" "A C" "A B"
> get.string(ng)
[1] "A B A C A B B"
> get.nextwords(ng)
Error in .local(ng, ...) : Not yet implemented
\end{lstlisting}
